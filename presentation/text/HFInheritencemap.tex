\begin{tikzpicture}[
    >={Latex[width=2mm,length=2mm]},
        base/.style = {rectangle, rounded corners, draw=black,
                    minimum width=2cm, minimum height=0.5cm, text
                    centered, font=\sffamily},
        basecode/.style = {rectangle, rounded corners, draw=black,
                    minimum width=2cm, minimum height=0.5cm, text
                    centered, font=\sffamily, align=left},
        activityStarts/.style = {base, fill=blue!30, drop shadow},
        startstop/.style = {base, fill=red!25, drop shadow},
        startstopcode/.style = {basecode, fill=red!25, drop shadow},
        activityRuns/.style = {base, fill=green!25, drop shadow},
        process/.style = {base, fill=white!15, font=\sffamily, drop shadow},
        processcode/.style = {basecode, fill=white!15, font=\sffamily, drop shadow},
    scale=0.8, 
    node distance=1.5cm, 
    every node/.style={fill=white, font=\sffamily, scale=0.7},
    align=center]
    \node (Hexpander) [processcode] {
        \hltexttt{Hexpander}
    };
    \node (GaussianQuadrature) [processcode, left of=Hexpander, xshift=-3cm] {
        \hltexttt{GaussianQuadrature}
    };
    \node (Cartesian) [processcode, below of=HartreeFockSolver] {
        \hltexttt{Cartesian}
    };
    \node (Hermite) [processcode, below of=Cartesian] {
        \hltexttt{Hermite}
    };
    \node (HartreeFockSolver) [processcode, below of=Hermite] {
        \hltexttt{HartreeFockSolver}
    };
    \node (GaussianIntegrals) [processcode, above right of=HartreeFockSolver, xshift=4cm] {
        \hltexttt{GaussianIntegrals}
    };
    \node (DoubleWell) [processcode, below right of=HartreeFockSolver, xshift=4cm] {
        \hltexttt{DoubleWell}
    };
    \node (DWC) [processcode, below of=HartreeFockSolver] {
        \hltexttt{DWC}
    };
    \draw[->] (GaussianQuadrature) -- (Hexpander);
    \draw[->] (Hexpander) to [out=0, in=90] (GaussianIntegrals);
    \draw[->] (Cartesian) to [out=0, in=90] (GaussianIntegrals);
    \draw[->] (Hermite) to [out=0, in=90] (GaussianIntegrals);
    \draw[->] (HartreeFockSolver) to [out=0, in=180] (GaussianIntegrals);
    \draw[->] (HartreeFockSolver) to [out=0, in=180] (DoubleWell);
    \draw[->] (DWC) to [out=0, in=180] (DoubleWell);
    \draw[->] (GaussianIntegrals) -- (DoubleWell);
\end{tikzpicture}
