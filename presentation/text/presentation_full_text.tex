\documentclass[10pt]{beamer}

\usepackage[utf8]{inputenc}
\usepackage[T1]{fontenc}
\usepackage[style=numeric-comp, 
            backend=biber,
            url=false,
            doi=true,
            eprint=false]{biblatex}
\usepackage{subcaption}
\usepackage{graphicx}
\usepackage{apacite}
\usepackage{amsmath}
\usepackage{siunitx}
\usepackage{braket}
\usepackage{physics}

\usetheme[progressbar=frametitle]{metropolis}
\usepackage{appendixnumberbeamer}

\usepackage{booktabs}
\usepackage[scale=2]{ccicons}

\usepackage{pgfplots}
\usepgfplotslibrary{dateplot}

\usepackage{xspace}
\newcommand{\themename}{\textbf{\textsc{metropolis}}\xspace}

\title{Quantum Many-Body Simulations of Double Dot System}
% \date{\today}
\date{}
\author{Alocias Mariadason}
\institute{Institute of Physics}
% \titlegraphic{\hfill\includegraphics[height=1.5cm]{logo.pdf}}

\begin{document}

\maketitle

\begin{frame}{Table of contents}
  \setbeamertemplate{section in toc}[sections numbered]
  \tableofcontents[hideallsubsections]
\end{frame}

\section{Introduction}

\begin{frame}[fragile]{Quantum-Dot}
    What is a quantum-dot?
        - small semiconductor nanostructure that confines motion of conduction.
        - Can be made with i.e electrostatic potentials.
        - $2$-$10$ nanometers in size
\end{frame}

\begin{frame}[fragile]{Quantum-Dot Model}
    Schrödinger equation $\mathcal{H}\ket{\psi} = E\ket{\psi}$
        - We model the quantum-dot as a many-body quantum system, meaning we solve the Schrödinger equation.
        - Let us express the hamiltonian
            \begin{itemize}
                \item $\mathcal{H} = - \frac{1}{2} \sum\limits_i \nabla^2_i +
                    \sum_{i<j} f\left(\blds{r}_j, \blds{r}_j\right) -
                    \frac{1}{2} \sum\limits_k \frac{\nabla^2_k}{M_k} + \sum\limits_{k<l}
                    g\left(\blds{R}_k,\blds{R}_l\right) +
                    V\left(\blds{R},\blds{r}\right)$
            \end{itemize}
            - This is the full system with a set of nuclei given here with an
            interaction potential $g$, kinetic for electrons, kinetic for
            nuclei an electron-electron interaction potential and lastly a
            confinement potential.
        - Born-Oppenheimer Approximation
            - Firstly we assume the nuclei to be far heavier than the electrons
            which means the system can be viewed as electrons moving in a
            vicinity around stationary nuclei.
            - The kinetic term then vanishes and the nuclei-interaction is a
            constant which can be ignored since adding a constant to the
            potential only alters the eigenvalues by a constant factor, the
            eigenfunctions remain unchanged.
            - We still need to model the electron-electron interaction and the
            confinement potential
\end{frame}

\begin{frame}[fragile]{Quantum-Dot Model}
    How do we model them?
        - Interaction: is the Coulomb repulsion
        - And for the confinement we studied two ways from >> ref <<. One is a simple parabolic-dot modelled by a harmonic oscillator.
        - Other is a double-dot, which is just a harmonic oscillator displaced in the $x$-direction.
        - Notice that for the confinement we have no terms including the nuclei. This is one of the characteristics of quantum-dot systems, we dont just set the nuclei interaction to constant, we also effectively ignore the entire thing (#nuclei have feeling)
\end{frame}

\section{Methods}
- In order to solve the many-body time-independent Schrödinger equation we need
some methods. We have used two method,
\begin{frame}[fragile]{Methods}
    Hartree-Fock(HF)
    Variational Monte-Carlo(VMC)
    - Before we explain the Hartree-Fock method it is worth-while to mention the Variational principle
\end{frame}

\begin{frame}[fragile]{Methods: Variational Principle}
    - It states this
    \begin{equation*}
        E_0 \leq \ecp{\mathcal{H}} =
        \frac{\Braket{\Psi|\mathcal{H}|\Psi}}{\Braket{\Psi|\Psi}}
        \label{eq:varPrinc}
    \end{equation*}
    - It basically says that given a system described by some Hamiltonian $\mathcal{H}$ the expectation value of the energy can NEVER be lower than the true ground-state energy of the system $E_0$. This might seem trivial, but it gives us the oppurtunity to define a trial wavefunction $\Psi$ for the the system and then minimize this. This simple idea is the main fundament on which both the Hartree-Fock method and the Variational Monte-Carlo method rests on.
\end{frame}

\begin{frame}[standout]{Methods: Slater Determinant And Energy Functional}
    Pauli Principle
        - The Pauli principle states that for systems of identical particles no same particle can occupy the same quantum state simultaneously.
    Slater Determinant
        - Following this result the many-body wavefunction for a fermionic
        system to be anti-symmetric and represented by a Slater determinant.
        - The P-operator indicates permuations over all possible $\psi_i$ and
        the latter case is the symmetric case present for bosons.
        - Taking the anti-summetric wavefunction as the ansatz for our
        trial-wavefunction we end up with the following energy-functional
        - The H-zero term is just the single-body Hamiltonian
        - This is the start of the Hartree-Fock method.
\end{frame}

\begin{frame}[standout]{Methods: Hartree-Fock}
    Assumptions
        - The first is already assumed and explained
        - We also restrict the system to not be relativistic. This is true for the low number of particles we work with.
        - The single-slater wavefunction assumption holds as long as we work
        with closed-shell systems. This means that the number of particles
        always fulfills the degeneracy levels, for the HO we have this figure.
        So we we can only run with $N=2,6,12,20$ and so on for $2D$ and
        $N=2,8,20$ for $3D$ for HO-system.
        - An for the MFA, this essentially means that the entire interaction
        potential can be described as a mean potential rather than specific
        individual point-potentials for each particle. This is fulfilled with
        the Coulomb interaction.
\end{frame}

\begin{frame}[standout]{Methods: Hartree-Fock}
    Contrained Minimization
        - As mentioned we need to minimize the energy. We notice one
        constraint, the spin orthogonality. We can minimize with the Lagrange
        multiplier method with the following Lagrangian. The delta-E part is
        the energy variation we seek the zero of.
        - We then define a Fock-operator giving the eigenvalue equation, and
        two operators defined from the energy-functional.
        - We then have $N+1$ equations to be solved
\end{frame}

\begin{frame}[fragile]{Methods: Hartree-Fock}
    Integrate out spin
        - The $\psi$'s have till now all been spin-orbitals. That is a
        wavefunction consisting of a spacial function multiplied by a spin
        function. It is convenient to integrate the spin dependency out of the
        eigenvalue equation. There exists two ways of achieving this.
    Pair spins as
        - where $\phi$ is the spacial part and $\alpha$ and $\beta$ are two
        spin configurations, up and down.
        - We restrict both the spin-configurations to have the same spacial
        function.
    Expand
        - and expand the spacial part in some known basis $\xi$ (i.e HO)
    Roothan-Hall
        - and end up with the Roothan-Hall equation with the elements of the
        Fock-matrix as this, the sum now only goes up-to $N/2$ because of the
        spin pairing
    Poople-Nesbet
        - If we dont restrict the spin-orbitals we end up with two Roothan-Hall
        equations and a Fock-Matrix with mixed elements.
    - The equations are a set of non-linear equations and have to be solvet
    iteratively until a convergence is reached.
\end{frame}

\begin{frame}[fragile]{Methods: Hartree-Fock}
    - The algorithm is to first pre-build the two-body, one-body and overlap
    matrices. The initial coefficients can be set to unit-matrix if nothing
    else is known. Set the Fock matrix either the restricted or the
    unrestricted form. Perform an optional mixing of $F$. Solve the generalized
    eigenvalue problem, calculate density matrix, mix this if desired, check
    for convergence. If reached calculate energy and end, if not save
    eigenvalues and density matrix (for mixing).
\end{frame}

\begin{frame}[fragile]{Methods: Variational Monte-Carlo}
    Variational Principle
        - Again we make use of the variational principle
        - However the variational method applies a statistical approach to the
        problem, we first rewrite the expectation value with a so-called local
        energy defined as this. The form of the integral is recognized as an
        integral over the probability distribution $\abs{\Psi}^2$.
   Metropolis-Hastings
        - We can solve this with the Metropolis algorithm in which we choose
        positions $\blds{R}$ randomly and accept accoring to this
        - The $T$-ratio is defined by Importance Sampling. We will skip the
        derivation, but the main idea is to use the fact that the quantum
        mechanical system can be modelled as an isotropic diffusion process,
        solve the Focker-Plank and Langevin equations and end up with a new
        proposal for the positions defined by a quantum force which pushes the
        particles towards regions where the wavefunction is large. $\xi$ is a
        normal distributed number with mean $0$ and unit variance
\end{frame}

\begin{frame}[fragile]{Methods: Variational Monte-Carlo}
    Algortihm (figure)
        - Initialize state, i.e set positions and calculate wavefunction,
        gradient, laplacian, local energy etc. Pick a random state accoring to
        the acceptance definition, either accept or reject. If we have enough
        samples, end, and if not keep accumulating.
\end{frame}

\section{Wavefunction}
    - We have now seen the Hartree-Fock and Variational Monte-Carlo methods,
    but one part of both has been ignored. Namely the choice of the
    wavefunction.

\begin{frame}[standout]{Methods: Wavefunction}
    - For the Hartree-Fock method we had the following
\end{frame}

\begin{frame}[standout]{Methods: Wavefunction}
    - We need to make a choice for this (focus red $\xi$)
\end{frame}

\begin{frame}[standout]{Wavefunction: Integral Elements}
    - The overlap, potential, kinetic and interaction integrals. We need to
    make a choice for the basisfunctions in order to find an analytic solution
    to these integrals.
\end{frame}

\begin{frame}[fragile]{Wavefunction: Single-Well}
    - One choice for the single-well case is the Hermite-functions which form
    an eigenbasis for the quantum harmonic oscillator system.
    - A solution in polar-coordinates already exists, however our goal is to
    actually use the Hartree-Fock basis in the variational method. The full
    wavefunction has to be evaluated in that case and the wavefunction in polar
    has a complex part which complicated the calculations.
    - In order to avoid that we went full Cartesian instead. Unfortunatly a
    solution did not exist, so we calculated it by using the fact that
    Hermite-Functions consist of Hermite-Gaussian constituents, with $C$ being the
    hermite coefficients.
    - The expressions to be solved are thus integrals over Hermite-Gaussian
    functions. A solution in $3D$ already exists for such integrals, and we
    used a similar approach and found the solution to be
\end{frame}

\begin{frame}[fragile]{Wavefunction: Single-Well Integral Elements}
    - We have a recursion relation for the coefficient $E$ and integrals $\xi$,
    $\zeta$ integrals have to be solved numerically and can be solved accuratly
    with Gaussian-Quadrature. In particular the $2D$ case follows the form of
    Gauss-Chebyshev quadrature. Again not to important to understand this
    fully, it is presented to show that solving the integral over
    Hermite-Functions is possible in Cartesian coordinates.
    - This is all that is needed for the Hartree-Fock method for the
    single-well case. For the double-well we still need to build a basis.
\end{frame}

\begin{frame}[fragile]{Wavefunction: Double-Well}
    - Notice that double-well is only a perturbation of single-well
    - Reasonable to use HO-functions as basis in an expansion
    - Projecting with the HO-bra we get an eigenvalue equation to be solved for the coefficients
    - And we have the following integral-elements. The eigenvalues DW are
    readily available from the solution to the eigenvalue equation and the
    two-body integral elements are also ready from the full solution of the
    harmonic oscillator system, meaning we only need to assemble the elements for Hartree-Fock.
\end{frame}

\section{Implementation}
\section{Summary and Conclusion}

{\setbeamercolor{palette primary}{fg=black, bg=white}
\begin{frame}[standout]
  Questions?
\end{frame}
}

\appendix

\begin{frame}[fragile]{Questions}
\end{frame}

\begin{frame}[allowframebreaks]{References}

  \bibliography{demo}
  \bibliographystyle{abbrv}

\end{frame}

\end{document}
