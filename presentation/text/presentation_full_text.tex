\documentclass[10pt]{beamer}

\usepackage[utf8]{inputenc}
\usepackage[T1]{fontenc}
\usepackage[style=numeric-comp, 
            backend=biber,
            url=false,
            doi=true,
            eprint=false]{biblatex}

\usetheme[progressbar=frametitle]{metropolis}
\usepackage{appendixnumberbeamer}

\usepackage{booktabs}
\usepackage[scale=2]{ccicons}

\usepackage{pgfplots}
\usepgfplotslibrary{dateplot}

\usepackage{xspace}
\newcommand{\themename}{\textbf{\textsc{metropolis}}\xspace}

\title{Quantum Many-Body Simulations of Double Dot System}
% \date{\today}
\date{}
\author{Alocias Mariadason}
\institute{Institute of Physics}
% \titlegraphic{\hfill\includegraphics[height=1.5cm]{logo.pdf}}

\begin{document}

\maketitle

\begin{frame}{Table of contents}
  \setbeamertemplate{section in toc}[sections numbered]
  \tableofcontents[hideallsubsections]
\end{frame}

\section{Introduction}

\begin{frame}[fragile]{Quantum-Dot}
    What is a quantum-dot?
        - small semiconductor nanostructure that confines motion of conduction.
        - Can be made with i.e electrostatic potentials.
        - $2$-$10$ nanometers in size
\end{frame}

\begin{frame}[fragile]{Quantum-Dot Model}
    Schrödinger equation $\mathcal{H}\ket{\psi} = E\ket{\psi}$
        - We model the quantum-dot as a many-body quantum system, meaning we solve the Schrödinger equation.
        - Let us express the hamiltonian
            \begin{itemize}
                \item $\mathcal{H} = - \frac{1}{2} \sum\limits_i \nabla^2_i +
                    \sum_{i<j} f\left(\blds{r}_j, \blds{r}_j\right) -
                    \frac{1}{2} \sum\limits_k \frac{\nabla^2_k}{M_k} + \sum\limits_{k<l}
                    g\left(\blds{R}_k,\blds{R}_l\right) +
                    V\left(\blds{R},\blds{r}\right)$
            \end{itemize}
            - This is the full system with a set of nuclei given here with an
            interaction potential $g$, kinetic for electrons, kinetic for
            nuclei an electron-electron interaction potential and lastly a
            confinement potential.
        - Born-Oppenheimer Approximation
            - Firstly we assume the nuclei to be far heavier than the electrons
            which means the system can be viewed as electrons moving in a
            vicinity around stationary nuclei.
            - The kinetic term then vanishes and the nuclei-interaction is a
            constant which can be ignored since adding a constant to the
            potential only alters the eigenvalues by a constant factor, the
            eigenfunctions remain unchanged.
            - We still need to model the electron-electron interaction and the
            confinement potential
\end{frame}

\begin{frame}[fragile]{Quantum-Dot Model}
    How do we model them?
        - Interaction: is the Coulomb repulsion
        - And for the confinement we studied two ways from >> ref <<. One is a simple parabolic-dot modelled by a harmonic oscillator.
        - Other is a double-dot, which is just a harmonic oscillator displaced in the $x$-direction.
        - Notice that for the confinement we have no terms including the nuclei. This is one of the characteristics of quantum-dot systems, we dont just set the nuclei interaction to constant, we also effectively ignore the entire thing (#nuclei have feeling)
\end{frame}

\section{Methods}
- In order to solve the many-body time-independent Schrödinger equation we need
some methods. We have used two method,
\begin{frame}[fragile]{Methods}
    - Hartree-Fock(HF)
    - Variational Monte-Carlo(VMC)
\end{frame}

\section{Implementation}
\section{Summary and Conclusion}

{\setbeamercolor{palette primary}{fg=black, bg=white}
\begin{frame}[standout]
  Questions?
\end{frame}
}

\appendix

\begin{frame}[fragile]{Questions}
\end{frame}

\begin{frame}[allowframebreaks]{References}

  \bibliography{demo}
  \bibliographystyle{abbrv}

\end{frame}

\end{document}
