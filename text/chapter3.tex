%%%%%%%%%%%%%%%%%%%%%%%%%%%%%%%%%% Chapter 3 %%%%%%%%%%%%%%%%%%%%%%%%%%%%%%%%%%
\chapter{Many-Body Quantum Theory\label{chapter:3}}     
    This chapter takes forth the theory regarding the basics of identical
    particles and \txtit{many-body quantum mechanics}. The reader is referred
    to \cite{GriffQuan} for an introductory text on quantum mechanics(for
    single particles) and also the so-called \txtit{Dirac-notation} used
    throughout the entire chapter.

\section{The Hamiltonian and the Born-Oppenheimer Approximation\label{sec:3.1}}
    The task at hand is to solve the many-body system described by
    \txtit{Schrödinger's} equation
        \begin{equation}
            H\ket{\Psi_i} = E_i\ket{\Psi_i}
            \label{eq:SE}
        \end{equation}
    for some state $\ket{\Psi_i}$ with energy $E_i$. Usually the desired state
    is the ground-state energy $E_0$ of the system meaning we are primarily
    interested in the \txtit{ground-state} $\ket{\Psi_0}$. \\ With the goal
    determined we can define the system to consist of $N$ identical
    particles\footnote{These are in both atomic physics and in the quantum dot
    case always fermions or bosons.} with positions
    $\{\blds{r}_i\}^{N-1}_{i=0}$ and $A$ nuclei with positions
    $\{\blds{R}_k\}^{A-1}_{k=0}$. The Hamiltonian $H$ is then
        \begin{equation}
            H = - \frac{1}{2} \sum\limits_i \nabla^2_i + \sum_{i<j}
            f\left(\blds{r}_j, \blds{r}_j\right) - \frac{1}{2} \sum_k
            \frac{\nabla^2_k}{M_k} + \sum_{k<l}
            g\left(\blds{R}_k,\blds{R}_l\right) +
            V\left(\blds{R},\blds{r}\right)
        \end{equation}
    The first and second terms represent the kinetic- and inter-particle
    interaction terms\footnote{This is usually the well-known Coulomb
    interaction.} for the $N$ identical particles while the latter three
    represent kinetic- and interaction terms for the nuclei(with the last one
    being the nuclei-particle interaction). The constant $M_k$ is the mass of
    nucleon $k$ and $Z_k$ is the corresponding atomic number.

    We assume the nuclei to be much heavier than the identical particles,
    meaning they move much slower, at which the system can be viewed as
    electrons moving around the vicinity of stationary nuclei.  This means the
    kinetic term for the nuclei vanish and the nuclei-nuclei interaction
    becomes a constant\footnote{Adding a constant to an operator does not alter
    the eigenvector, only the eigenvalues by the constant
    factor\cite{linalgDavid}.}. The approximation we end up with is the
    so-called \txtit{Born-Oppenheimer approximation} and the Hamiltonian is now
        \begin{equation}
            H = H_0 + H_I
        \end{equation}
    where we have split the Hamiltonian in a \txtit{one-body} part and a
    \txtit{two-body} or \txtit{interaction} parts defined as
        \begin{equation}
            H_0 \equiv - \frac{1}{2} \sum\limits_i \nabla^2_i +
            V\left(\blds{R},\blds{r}\right)
        \end{equation}
    and
        \begin{equation}
            H_I \equiv \sum_{i<j} f\left(\blds{r}_j, \blds{r}_j\right)
        \end{equation}

\section{Slater Determinant and Permanent}
    Throughout \Arf{sec:3.1} we only referred to the wavefunction $\Psi$ as a
    state, a function closely connected to the probabilistic nature of the
    quantum particles. However, we have not given it a form. One possible
    solution is the \txtit{Hartree product} $\Psi_{\text{H}}$ defined as
        \begin{equation}
            \Psi_{\text{H}} = \prod_i\psi_i(\blds{r}_i)
        \end{equation}
    with $\{\psi\}_{i=0}^N$ being the orbitals which solve the single-particle
    Schrödinger equation for $H_0$. The Hartree-product is unfortunately a poor
    choice since it does not solve the $H_I$ part meaning it is not a
    physically valid solution. This comes from the fact that the
    Hartree-product does not take into account the fact that the particles in
    question are \txtit{identical particles}. Since the particles are
    identical, switching the labels on the particles shouldn't change the
    expectation value of some observable. If we run this observation through we
    end up with the conclusion that the state $\ket{\Psi}$ must be either
    symmetric or antisymmetric with the symmetric part being the \txtit{bosonic
    state} and antisymmetric being the \txtit{fermionic state}. The connection
    between antisymmetric states and fermions is called the \txtit{Pauli
    exclusion principle}.

    The problem with the Hartree-product is with the above sentiment, that is
    not symmetric nor antisymmetric. However we can transform it with an
    operator
        \begin{equation}
            \mathcal{B} \equiv \frac{1}{N!}\sum_P\sigma_bP
        \end{equation}
    where $\sigma_b$ is a sign operator which is just $1$ for the symmetric
    case and $(-1)^p$ for the antisymmetric case, $P$ is a permutation operator
    that switches the labels on particles
    \footnote{$P\psi_i\psi_j=\psi_j\psi_i$} and $p$ is the parity of
    permutations. The solution $\Psi_T$ to the Schrödinger equation can now be
    written as
        \begin{equation}
            \Psi_T\left(\blds{r}\right) =
            \sqrt{N!}\mathcal{B}\Psi_{\text{H}}\left(\blds{r}\right)
            \label{eq:psiOpB}
        \end{equation}
    The antisymmetric case of $\mathcal{B}$ results in a \txtit{Slater
    determinant} 
        \begin{equation}
            \Psi^{\text{AS}}_T = \frac{1}{\sqrt{N!}}\sum_{P}(-1)^pP\prod_i\psi_i
        \end{equation}
    while the symmetric case gives the so-called \txtit{permanent}\footnote{The
    permanent is basically just a determinant with all the negative signs
    replaced by positive ones.}.
        \begin{equation}
            \Psi^{\text{S}}_T = \frac{1}{\sqrt{N!}}\sum_{P}P\prod_i\psi_i
        \end{equation}

\section{Variational Principle\label{sec:varPrinc}}
    One important remark is that the Slater determinant and the permanent do
    not solve the interaction part, but only serves as a so-called
    \txtit{ansatz} or guess on the true ground-state wavefunction. This is
    quite useful due to the \txtit{variational principle}. \\
    The Variational principle states that for any normalized function $\Psi$ in
    Hilbert Space >> REF HILBERT << with a Hermitian operator $H$ the minimum
    eigenvalue $E_0$ for $H$ has an upper-bound given by the expectation value
    of $H$ in the function $\Psi$. That is
        \begin{equation}
            E_0 \leq \ecp{H} = \bra{\Psi}H\ket{\Psi} = \int \Psi^{*}H\Psi \md r
            \label{eq:varPrinc}
        \end{equation}
    See \cite{GriffQuan} for proof and more.

    The mentioned ansatz is thereby guaranteed to give energies larger than or
    equal the true ground state energy meaning a minimization method is
    sufficient in order to get closer to this minimum. 
   
\section{Energy Functional\label{sec:energyFunc}}
    We can find a more convenient expression for this energy by using
    \Arf{eq:psiOpB, eq:varPrinc}. This gives us
        \begin{equation}
            E_0 = N!\bra{\Psi_{\text{H}}}H\mathcal{B}\ket{\Psi_{\text{H}}}
        \end{equation}
    where the hermitian and unitary property of $\mathcal{B}$ as well as the
    fact that $\mathcal{B}$ and $H$ commute have been used.
