%%%%%%%%%%%%%%%%%%%%%%%%%%%%%%%%%% Chapter 3 %%%%%%%%%%%%%%%%%%%%%%%%%%%%%%%%%%
\chapter{Many-Body Quantum Theory\label{chapter:3}}     
    This chapter takes forth the theory regarding the basics of identical
    particles and \txtit{many-body quantum mechanics}. The reader is referred
    to \cite{GriffQuan} for an introductory text on quantum mechanics(for
    single particles) and also the so-called \txtit{Dirac-notation} used
    throughout the entire chapter.

\section{The Hamiltonian and the Born-Oppenheimer Approximation}
    The task at hand is to solve the many-body system described by
    \txtit{Schrödinger's} equation
        \begin{equation}
            H\ket{\Psi_i} = E_i\ket{\Psi_i}
            \label{eq:SE}
        \end{equation}
    for some state $\ket{\Psi_i}$ with energy $E_i$. Usually the desired state
    is the ground-state energy $E_0$ of the system meaning we are primarily
    interested in the \txtit{ground-state} $\ket{\Psi_0}$. \\ With the goal
    determined we can define the system to consist of $N$ identical
    particles\footnote{These are in both atomic physics and in the quantum dot
    case always fermions or bosons.} with positions
    $\{\blds{r}_i\}^{N-1}_{i=0}$ and $A$ nuclei with positions
    $\{\blds{R}_k\}^{A-1}_{k=0}$. The Hamiltonian $H$ is then
        \begin{equation}
            H = - \frac{1}{2} \sum\limits_i \nabla^2_i + \sum_{i<j}
            f\left(\blds{r}_j, \blds{r}_j\right) - \frac{1}{2} \sum_k
            \frac{\nabla^2_k}{M_k} + \sum_{k<l}
            g\left(\blds{R}_k,\blds{R}_l\right) +
            V\left(\blds{R},\blds{r}\right)
        \end{equation}
    The first and second terms represent the kinetic- and inter-particle
    interaction terms\footnote{This is usually the well-known Coulomb
    interaction.} for the $N$ identical particles while the latter three
    represent kinetic- and interaction terms for the nuclei(with the last one
    being the nuclei-particle interaction). The constant $M_k$ is the mass of
    nucleon $k$ and $Z_k$ is the corresponding atomic number.

    We assume the nuclei to be much heavier than the identical particles,
    meaning they move much slower, at which the system can be viewed as
    electrons moving around the vicinity of stationary nuclei.  This means the
    kinetic term for the nuclei vanish and the nuclei-nuclei interaction
    becomes a constant\footnote{Adding a constant to an operator does not alter
    the eigenvector, only the eigenvalues by the constant
    factor\cite{linalgDavid}.}. The approximation we end up with is the
    so-called \txtit{Born-Oppenheimer approximation} and the Hamiltonian is now
        \begin{equation}
            H = H_0 + H_I
        \end{equation}
    where we have split the Hamiltonian in a \txtit{one-body} part and a
    \txtit{two-body} or \txtit{interaction} parts defined as
        \begin{equation}
            H_0 \equiv - \frac{1}{2} \sum\limits_i \nabla^2_i +
            V\left(\blds{R},\blds{r}\right)
        \end{equation}
    and
        \begin{equation}
            H_I \equiv \sum_{i<j} f\left(\blds{r}_j, \blds{r}_j\right)
        \end{equation}
