%%%%%%%%%%%%%%%%%%%%%%%%%%%%%%%%%% Chapter 6 %%%%%%%%%%%%%%%%%%%%%%%%%%%%%%%%%%
\chapter{Conclusion \label{chapter:7}}
    The aim of the thesis was to build a basis for closed-shell harmonic
    oscillator and double-well potentials in both two- and three dimensions and
    make a general code for simulating said potentials using the Hartree-Fock
    method and the Variational Monte Carlo method. In order to simulate the
    double-well potential we had to build the basis ourselves. This was made
    using the harmonic oscillator functions as basis functions. To run the
    Hartree-Fock simulation for the double-well a set of integral elements had
    to be calculated. This was successfully done in Cartesian coordinates from
    scratch.

    The implementation of the Hartree-Fock method and the variational method
    were both made in \CC and can be extended with little effort to other
    systems. With the extension being all the simpler for systems with basis
    functions that are Hermite-functions or Gauss-Hermite functions. The
    implementation was also tested and verified with the harmonic oscillator
    system and the tests perform very well indicating that the machinery is
    working well. The Hartree-Fock implementation reproduces the Hartree-Fock
    limit energies compared with literature for up to $30$ particles in the two
    dimensional case and $20$ particles in three dimensions. The energies for
    larger number of particles converge towards the limit indicating that we
    only need to run with more shells in order to reach it.  We also
    experimented with different Jastrow factors with the variational method and
    managed to reproduce the energies from the literature for those as as well
    and managed to get a lower energy than with the Hartree-Fock method as
    expected.

    The double-well simulation also seem good. The energies are lower than the
    corresponding energy with the single-well and the variational method
    manages to lower the energies further from the Hartree-Fock limit. The
    double-well was also run for more than $2$ particles. The Hartree-Fock
    limit was reached for all the cases with the double-well.

    We made an effort in automation of the numerical minimization in the
    variational method. This was fairly successful as long as the Simulated
    Annealing scheme was run for a long time. Still the minimization had to be
    tweaked and monitored in order to get a good result. The methods themselves
    were however successful within their own limitations and worked as a great
    tools for finding the ground-state energy.

    In general the project was a success. The methods of interest performed
    well for the systems to be studied and the resulting code from the
    implementations ended up general and extendable.

    We limited ourselves to isotropic gaussian functions (the constituents of
    Hermite-functions), however it is reasonable to imagine that non-isotropic
    gaussians could give an equally good or even better basis for the
    double-well system. As it is now the computational time is not too long,
    however it could be reduced since one might need far less basis functions
    when using non-isotropic gaussians. Experimentations with the centering of
    the gaussians is also a part left out in this thesis. The two-body elements
    in Hartre-Fock are setup in the simplest way by writing out all the
    elements. This matrix is known to be very sparse meaning an implementation
    using sparse matrices would give a better performance.
