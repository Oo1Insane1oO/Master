\thispagestyle{empty}
\begin{center} \vspace{1cm}
    \textbf{\Large{\mtitle}}\\ \vspace{0.5cm}
    \small{by}\\ \vspace{0.5cm}
    \large{\mauthor}\\ \vspace{4.4cm}
    \large{THESIS}\\ \vspace{0.3cm}
    \small{for the degree of}\\ \vspace{0.3cm}
    \large{MASTER OF SCIENCE}\\ \vspace{0.7cm}
    \includegraphics[scale=1.0]{text/UiO_Segl_pms485.eps} \\ \vspace{0.5cm}
    \large{Faculty of Mathematics and Natural Sciences \\ University of Oslo} \\ \vspace{0.5cm}
    \small{\mdate}\\ \vfill
\end{center}

\thispagestyle{empty}
\clearpage
TODO: write creative commons licence
\thispagestyle{empty}
\clearpage

\begin{center}
    \textbf{\Large{Abstract}}\\ \vspace{0.6cm}
\end{center}
    We have implemented two different quantum many-body method: Hartree-Fock
    and Variational Monte Carlo. The system studies were closed-shell quantum
    dot systems with the single-well harmonic oscillator potential and the
    double-well potential. The basis used consisted of harmonic oscillator
    functions and calculations were performed in cartesian coordinates. The
    theory is presented in detail including the calculation of the two-body
    elements in the Hartree-Fock method. The resulting basis from the
    Hartree-Fock method was then used as function in the Slater determinant for
    the variational Monte Carlo method. The Hartree-Fock limit was reached for
    $30$ particles in two dimensions and $20$ in three for the single-well.
    Said limit was reached for $2$ up to $12$ particles for the two dimensional
    double-well and for $2$ up to $14$ for the three dimensional double-well.

\thispagestyle{empty}
\clearpage

\thispagestyle{empty}
\clearpage

\begin{center}
    \textbf{\Large{Acknowledgements}}\\ \vspace{0.6cm}
\end{center}
    I would like to firstly thank my awesome supervisor Morten Hjort-Jensen.
    Over the course of both my Master's thesis and throughout the bachelor you
    have always been enthusiastic and supportive of literaly anything I would
    throw at you. The support I got for my ideas and the immense freedom you
    gave me in my thesis truly made it enjoyable. Thank you for believing in me
    and keeping me motivated throughout the thesis.

    I would also like to thank Håkon Kristensen for helping me with everything
    in the thesis. Whitout your help I most certainly wouldn't have come
    through with the results presented. The help you gave me with the
    double-well problem and the ideas you have constantly given me have been
    the greatest help I could possible have had.

    The awesome figures made were all thanks to Anders Johansson who introduced
    me to Asymptote and helped me setup the initial script.

    I couldn't possibly finish without mentioning the rest of Computational
    Physics group with whom I've now spent 4 years with and enjoyed every bit
    of it. Everything from the game nights, the late night programming(and
    cursing...) to the endless houres spent eating lunch and playing Mario Cart
    and Super Smash has without doubt made my days as a student all the better.
\thispagestyle{empty}
\clearpage

{%
    \microtypesetup{protrusion=false}
    \tableofcontents
    \microtypesetup{protrusion=true}
    \thispagestyle{empty}
    \clearpage}%

\thispagestyle{empty}
\clearpage

\begin{center}
    \textbf{\Large{Symbols List}}\\ \vspace{0.6cm}
    Work in progress
    \begin{table}[H]
        \centering
        \begin{tabular}{rc}
            \textbf{Symbol} & \textbf{Meaning} \\
            $\psi$ & Spin-orbital \\
            $\phi$ & Spacial part of spin-orbital \\
            $\chi$ & Basis functions used in basis-expansion \\
            $\Psi$ & Trial wavefunction \\
            $\Phi$ & Slater determinant \\
            $g$ & Hermite-Gaussian \\
            $G$ & Gaussian-Type-Orbital (GTO) \\
            $\Omega$ & Overlap Distribution \\
            $\xi$ & Recurrence relation for Coulomb integral \\
            $\blds{x}$ & Bold lower case symbol is a \txtit{vector}. \\
            $\blds{X}$ & Bold upper case symbol is a \txtit{matrix}. \\
            $X_{ij}$ & Element $(i,j)$ of matrix $\blds{X}$. \\
            $\blds{x}_{ij}$ & Vector difference $\blds{x}_i - \blds{x}_j$. \\
            $x_{ij}$ & Length of vector difference $\blds{x}_{ij}$. \\
            $\nabla_x$ & Gradient with derivatives of components in $\blds{x}$ \\
            $\nabla^2_x$ & Laplacian with second derivatives of components in $\blds{x}$ \\
            $\mathcal{H}$ & Caligraphed symbol is an operator. \\
            $\{X\}_{k=0}^{N}$ & A set with elements $X_0,\dots,X_N$
        \end{tabular}
        \caption{List of symbols used with explaination.}
        \label{tab:symbols}
    \end{table}
\end{center}
\thispagestyle{empty}
\clearpage

\begin{center}
    \textbf{\Large{Source Code}}\\ \vspace{0.6cm}
    Work in progress
\end{center}
\thispagestyle{empty}
\clearpage
