%%%%%%%%%%%%%%%%%%%%%%%%%%%%%%%%%% Chapter 1 %%%%%%%%%%%%%%%%%%%%%%%%%%%%%%%%%%
\chapter{Introduction\label{chapter:1}}
    At present time we live in an era of flourishing technological advances and
    groundbreaking scientific discoveries. A time in which discoveries are made
    across fields with large cooperative work. With the scale of the problems,
    so to does the demand for computations in science grow.

    The modern computer was introduced to the world somewhere in the middle of
    the last century and has become a staple tool in scientific work ever
    since. Within all big breakthroughs in science in the modern days, some
    kind of computer simulations were made, a numerical experiment was made
    prior to the actual physical experiments. The reason for this lies in the
    heart of quantum mechanics, the \txtit{Schrödinger equation}. The time
    dependent version if this equation is
        \begin{equation}
            i\hbar \frac{\prtl}{\prtl t} \Psi\left(\blds{r},t\right) =
            \mathcal{H}\Psi\left(\blds{r},t\right),
        \end{equation}
    with initial conditions which can be \txtit{stationary states}. These are
    then solved by the \txtit{time-independent Schrödinger equation}
        \begin{equation}
            \mathcal{H}\Psi\left(\blds{r}\right) = E\Psi\left(\blds{r}\right)
        \end{equation}
    Analytical solutions to this equation does exist. However in the case the
    number of particles increases, meaning the number of parameters within
    $\blds{r}$ increases, analytical solutions become hard or even impossible
    to find and the good-old pen-and-paper approach dwindles in usefulness.
    The only possibility is then to resort to numerical solutions and the
    modern computer, although a beast to be tamed, is the best candidate tool.

\section{Many-Body Methods}
    In order to tackle the numerical problem that is solving the many-body
    Schrödinger equation, some of these are the first-order methods
    Hartree-Fock theory and Density Functional Theory, hierarchical methods
    such as Configuration Interaction and Coupled Cluster theory and
    statistical methods like Variational Monte Carlo and Diffusion Monte Carlo. 

    In case with Hartree-Fock and Density Functional Theory, the accuracy is
    not desirably high, however they are efficient and are most useful as
    inputs to the more accurate hierarchical methods\footnote{They are
    hierarchical in the sense that increasingly accurate approximations can be
    systematically constructed.} or the variational method.

    For the configuration interaction methods one either truncates the defined
    hierarchical structure\footnote{In which case it is actually known as
    Configurations Interaction.} or uses all contributions in which case the
    method is known as Full Configuration Interaction. These methods suffer
    from \txtit{exponential scaling} and are therefore much heavier in
    computational strain than the less accurate methods mentioned. The
    truncated version however is not size-consistent.  The truncated
    Coupled-Cluster method does not have any size-inconsistencies and achieves
    polynomial scaling making the Coupled-Cluster method a standard method for
    when high accuracy is in question. 

    The statistical methods mentioned employes a different approach by
    utilizing the statistical nature of the wavefunction in the Schrödinger
    equation and modeling the entire problem as a stochastic diffusion process.
    Although the accuracy is on par with the hierarchical methods, the
    strenuous effort involved in the minimization within the variational method
    can utterly obliterate ones spirit.

    One great property which statistical methods is that they readily, without
    much effort, gives the beasts known as \txtit{density-matrices} or $N$-body
    densities\footnote{Technically possible to obtain from the hierarchical
    methods as well, but it is harder.}. These can give some insight into
    desirable physics of the system. It is also worth to mention that the
    variational method is in many approaches an input to the more accurate
    diffusion method since the latter requires a good initial starting point.

    A certain system which techniques from many-body quantum theory can be used
    is with the \txtit{artificial atoms} known as \txtit{quantum dots}.
    Quantum dots were first proposed in the 50's and have since played a
    big part in some big breakthroughs in science and technology. They are
    essentially really small semiconductor systems and they have played a
    central role in nanophysics because of their nice electrical and optical
    properties. They tend to exist in solid state meaning they can be more
    easily cooled. The result is that experiments have been conducted with
    lasers, LEDS, the new generation of transistors in the modern computer and
    as the ultimate goal; to use quantum dots in logic gates in \txtit{quantum
    computers}. With this quantum-dots have become a popular and interesting
    quantum system to study and a possible route of tackling the problem is
    with a computational approach. The quantity of interest in the calculations
    is the \txtit{ground-state energy} which the lowest possible eigenvalue of
    the Hamiltonian.

\section{Structure and Goals\label{sec:structure_and_goals}}
    In this thesis we go into details with the popular Hartree-Fock method and
    the Variational Monte-Carlo method with the \txtit{artificial atoms} known
    as quantum dots by using single- and double-well potentials as primary
    systems of study with this chapter being a brief introduction to the
    structure and goals of the thesis.

    The main goal was to make a code-base for large-scale quantum many-body
    calculations from scratch. Of course there exists many such code-bases and
    competing with those in terms of scalability and computational efficiency
    is beyond the scope of this thesis, the choice for a from-scratch approach
    was made in order to gain some insight into the methods and a better
    understanding of the programming aspects which would follow. With this in
    mind, we did however get much inspiration from previous theses and
    implementations from other sources as well. \\
    The main goals for the thesis were
        \begin{enumerate}
            \item Make a general \CC code for the \txtit{Hartree-Fock method}
                and the \txtit{Variational Monte Carlo method} which could take
                any basis into play with minimal effort.
            \item Use the \CC code on different \txtit{quantum mechanical
                many-body systems} and use this as validation of the code and
                benchmark the code.
            \item Build an optimal one-body basis with the Hartree-Fock method
                and make improvements on this basis with the variational
                Monte-Carlo method with a \txtit{Slater-Jastrow} wavefunction.
        \end{enumerate}
    With this in mind a general open-source code for the restricted
    Hartree-Fock method was developed and is open for use in
    \url{https://github.com/Oo1Insane1oO/HartreeFock}. The variational
    Monte-Carlo method is also open-source and for use in
    \url{https://github.com/Oo1Insane1oO/VMC}. Both repositories each have a
    directory with tests implemented, see \Arf{sec:verification} for more
    information on these tests. \\
    Extending the code to other systems is made easier with python scripts
    given in the mentioned repositories. A better description of these are
    given in the repositories themselves and involve auto-generation of
    abstract wavefunction classes which only need to be filled in with
    analytical expressions and are otherwise already integrated with the
    existing code. \\
    The thesis itself is built in four parts
        \begin{itemize}
            \item Quantum Theory: Theory for the implementation.
            \item Choice of Basis: The form of the wavefunction.
            \item Implementation: The code with optimizations.
            \item Results: The resulting data from the simulations.
        \end{itemize}
    We basically start of with the background in quantum many-body theory, make
    a specialization into a set of basis functions, implement these in a
    general \CC-code with auto-generation of expressions and template classes
    in Python and present the results from the simulations run with the
    implementation.
