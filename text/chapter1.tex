%%%%%%%%%%%%%%%%%%%%%%%%%%%%%%%%%% Chapter 1 %%%%%%%%%%%%%%%%%%%%%%%%%%%%%%%%%%
\chapter{Introduction\label{chapter:1}}
    This chapter will give a brief introduction to the structure of the thesis
    and the goals laid forth beforehand.

\section{Structure and Goals\label{sec:structure_and_goals}}
    The main goal(at least at some point into the semester) was to make a
    code-base for large-scale quantum many-body calculations from scratch. Of
    course there exists many such code-bases and competing with those in terms
    of scalability and computational efficiency is beyond the scope of this
    thesis, the choice for a from-scratch approach was made in order to gain
    some insight into the methods and a better understanding for the
    programming aspects which would follow. With this in mind, we did however
    get much inspiration from previous theses and implementations from other
    sources as well. \\
    The main goals for the thesis was
        \begin{itemize}
            \item Make a general \CC code for the \txtit{Hartree-Fock method}
                and the \txtit{Variational Monte Carlo method} which could take
                any basis into play with minimal effort.
            \item Use the \CC code on different \txtit{quantum mechanical
                many-body systems} and use this as validation of the code and
                benchmark the code.
            \item Build an optimal one-body basis with the Hartree-Fock method
                and make improvements on this basis with the variational
                Monte-Carlo method with a \txtit{Slater-Jastrow} wavefunction.
        \end{itemize}
    With this in mind a general open-source code for the restricted
    Hartree-Fock method was developed and is open for use in
    \url{https://github.com/Oo1Insane1oO/HartreeFock}. The variational
    Monte-Carlo method is also open-source and for use in >> URL THIS <<. Both
    repositories each have a directory with tests implemented, see
    \Arf{sec:verification} for more information on these tests. \\
    Extending the code to other systems is made easier with python scripts
    given in the mentioned repositories. A better description of these are
    given in the repositories themselves and involve auto-generation of
    abstract wavefunction classes which only need to be filled in with
    analytical expressions and are otherwise already integrated with the
    existing code.
