%%%%%%%%%%%%%%%%%%%%%%%%%%%%%%%%%% Chapter 1 %%%%%%%%%%%%%%%%%%%%%%%%%%%%%%%%%%
\chapter{Introduction\label{chapter:1}}
    At present time we live in an era of flourishing technological advances and
    groundbreaking scientific discoveries. A time in which discoveries are made
    across fields with large cooperative work. With the scale of the problems,
    so to does the demand for computations in science grow.

    The modern computer was introduced to the world somewhere in the middle of
    the last century and has become a staple tool in scientific work ever
    since. Within all big breakthroughs in science in the modern days, some
    kind of computer simulations were made, a numerical experiment was made
    prior to the actual physical experiments. The reason for this? As problems
    get larger in scale, the possible closed-form solution with the good-old
    pen-and-paper approach dwindles. The only possibility is then to resort to
    numerical solutions and the modern computer, although a beast to be tamed,
    is the best candidate tool.

    Some of the earliest simulations made were indeed in the field of quantum
    many-body physics, the (in)famous Manhattan Project and ever since then the
    field of many-body quantum theory and quantum chemistry has grown rapidly
    and given promising results in relation to reality. It has been used to
    simulate atoms, bond-energies of atomic systems, studies of condensed
    matter physics and many more. 
    
    In this thesis we go into detail on the popular Hartree-Fock method and the
    Variational Monte-Carlo method with the quantum-dot single- and double-well
    potentials as primary systems of study with this chapter being a brief
    introduction to the structure and goals of the thesis.

\section{Structure and Goals\label{sec:structure_and_goals}}
    The main goal(at least at some point into the semester) was to make a
    code-base for large-scale quantum many-body calculations from scratch. Of
    course there exists many such code-bases and competing with those in terms
    of scalability and computational efficiency is beyond the scope of this
    thesis, the choice for a from-scratch approach was made in order to gain
    some insight into the methods and a better understanding for the
    programming aspects which would follow. With this in mind, we did however
    get much inspiration from previous theses and implementations from other
    sources as well. \\
    The main goals for the thesis was
        \begin{itemize}
            \item Make a general \CC code for the \txtit{Hartree-Fock method}
                and the \txtit{Variational Monte Carlo method} which could take
                any basis into play with minimal effort.
            \item Use the \CC code on different \txtit{quantum mechanical
                many-body systems} and use this as validation of the code and
                benchmark the code.
            \item Build an optimal one-body basis with the Hartree-Fock method
                and make improvements on this basis with the variational
                Monte-Carlo method with a \txtit{Slater-Jastrow} wavefunction.
        \end{itemize}
    With this in mind a general open-source code for the restricted
    Hartree-Fock method was developed and is open for use in
    \url{https://github.com/Oo1Insane1oO/HartreeFock}. The variational
    Monte-Carlo method is also open-source and for use in
    \url{https://github.com/Oo1Insane1oO/VMC}. Both repositories each have a
    directory with tests implemented, see \Arf{sec:verification} for more
    information on these tests. \\
    Extending the code to other systems is made easier with python scripts
    given in the mentioned repositories. A better description of these are
    given in the repositories themselves and involve auto-generation of
    abstract wavefunction classes which only need to be filled in with
    analytical expressions and are otherwise already integrated with the
    existing code. \\
    The thesis itself is built in three parts
        \begin{itemize}
            \item Quantum Theory: Theory for the implementation.
            \item Choice of Basis: The form of the wavefunction.
            \item Implementation: The code with optimizations.
            \item Results: The resulting data from the simulations.
        \end{itemize}
    We basically start of with the background in quantum many-body theory, make
    a specialization into a set of basis functions, implement these in a
    general \CC-code with auto-generation of expressions and template classes
    in Python and present the results from the simulations run with the
    implementation.
