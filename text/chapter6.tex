%%%%%%%%%%%%%%%%%%%%%%%%%%%%%%%%%% Chapter 6 %%%%%%%%%%%%%%%%%%%%%%%%%%%%%%%%%%
\chapter{Results \label{chapter:6}}
    No amount of physics theory will ever be interesting if it was not backed
    up by some proof or atleast some results. \\
    In this chapter we present the results from the simulations of the quantum
    dot system in the harmonic oscillator system and the double-well system. We
    also present some results regarding the minimization scheme in the VMC
    method.

\section{Tweaks and Experimentation}
    With the minimization methods clarified and the function to be minimized
    outlined the actual minimization could start\footnote{With the methods at
    disposal the \txtit{real challenge} was yet to come...}. \\
    Within the minimization methods presented there are quite the number of
    constant parameters which needed to be set pre-hand. Often one finds these
    some-what heuristically with a close eye on the function to be minimized.
    This approach is the one we used in the VMC method, with the most memorable
    one being the parameters set in the simulated annealing method.

\subsection{Simulated Annealing}
    With the simulated annealing the idea was to use it as a sort of
    "thermalization", that is to run it initially in hope that it finds the
    valley within the function-mesh in which the global minima lies. \\
    We tried anything from 

\section{Hartree-Fock}
    The results from the Hartree-Fock simulations of the single-well harmonic
    oscillator potential was
    \input{text/t1}
