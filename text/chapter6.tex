%%%%%%%%%%%%%%%%%%%%%%%%%%%%%%%%%% Chapter 6 %%%%%%%%%%%%%%%%%%%%%%%%%%%%%%%%%%
\chapter{Results \label{chapter:6}}
    No amount of physics theory will ever be interesting if it was not backed
    up by some proof or at least some results. \\
    In this chapter we present the results from the simulations of the quantum
    dot system in the harmonic oscillator system and the double-well system.
    The harmonic oscillator was used mainly as a benchmark for verification for
    which is performed well. We also present some results regarding the
    minimization scheme in the variational method. All the simulations with the
    variational method was done with importance sampling as well.

\section{Tweaks and Experimentation}
    With the minimization methods clarified and the function to be minimized
    outlined the actual minimization could start\footnote{With the methods at
    disposal the \txtit{real challenge} was yet to come...}. \\
    Within the minimization methods presented there are quite the number of
    constant parameters which needed to be set pre-hand. Often one finds these
    some-what manually with a close eye on the function to be minimized. This
    approach is the one we used in the VMC method, with the most memorable one
    being the parameters set in the simulated annealing method and the
    notorious step-size in the Metropolis sampling. 

\subsection{Hartree-Fock}
    With the Hartree-Fock method the only part tweaked was the number of
    previous quantities to use in the DIIS procedure. Essentially just changing
    $M$ in (\Arf{eq:errorDIISdef})
        \begin{equation}
            \blds{Y} = \suml{m=1}{M} c_m\blds{y}_i.
        \end{equation}
    The exact number was changed to get convergence, however somewhere between
    $M=2$ to $M=8$ gave good results.

\subsection{Variational Monte-Carlo}
    For the VMC runs the only parameter to be tweaked was the step-size $\Delta
    t$ which had to be tweaked depending on $\omega$ and the number of
    particles. The general rule was to increase $\Delta t$ when $\omega$ was
    lowered, this has to do with the form of the potential, see
    \Arf{subfig:HO1D}. For $N$, the number of particles, the step had to be
    slightly increased as $N$ increased, this is due to the repulsion part
    being stronger with more particles. \\
    We also mention that the only criteria for convergence was the absolute
    value of the derivative
        \begin{equation}
            \abs{\nabla_{\blds{\alpha}}\ecp{E\left[\blds{\alpha;\blds{R}}\right]}}
            < \epsilon
        \end{equation}
    With $\epsilon\in(0.1,0.001)$ depending on the size of the system. The
    general idea was to look for sign oscillations of the gradient with respect
    to the variational parameters. If it started to oscillate the maximum
    step-size in the quasi-newton scheme was reduced(the maximum allowed in the
    case with the Mor\'e-Thuente linesearch) and looked to see if any
    improvement was made.

\subsubsection{Simulated Annealing}
    With the simulated annealing the idea was to use it as a sort of
    "thermalization", that is to run it initially in hope that it finds the
    valley within the function-mesh in which the global minima lies. \\ 
    The approach was to run a great number of iterations starting with a
    temperature of $100$. In order to finish those runs in time they were run
    in parallel and the parameters that gave the lower energy and lowest where
    then used as a starting point for the quasi-newton method which was then
    run with a close eye on the energies and variance\footnote{Coined as the
    VBS method, VMC-babysitting.}.

\subsubsection{RBM}
    For the RBM-Jastrow we did not tweak as much manually, but we initialized
    the parameters in the RBM-wavefunction with a Gaussian distribution with
    mean $0$ and variance between $0.01-1.0$ and settled with a value that gave
    a fairly good acceptance rate and variance in the sampling without the
    energies being way off the results with the Pad\'e-function.

\section{Hartree-Fock}
\subsection{Harmonic Oscillator}
    \input{text/t1}
    \input{text/t3D}

\subsection{Double-Well\label{sec:dwbimp}}
    For the double-well we used the larges basis-set from the Hartree-Fock
    calculations, which was $120$ spacial functions. Meaning the sum over $l$
    in (\Arf{eq:dwexpansiondef}) 
        \begin{equation}
            \ket{\psiDW_p} = \sumll{l}C^{\text{DW}}_{lp}\ket{\psiHO_l},
        \end{equation}
    runs up to $120$ while the number of
    basisfunctions used is chosen just as usual\footnote{Keep increasing until
    Hatree-Fock limit is reached.}. Keep in mind that the degeneration is
    lifted in the double-well system. The resulting energies for are given in
    \Arf{tab:HODW2D, tab:HODW3D}.
    \input{text/t2DDW}
    \input{text/t3DDW}

\section{VMC}
    Here are the results for the quantum-dot simulations with the variational
    method both with and without a Hartre-Fock basis. We only used a basis as
    large as the one needed to reach the Hartree-Fock limit\footnote{See
    \Arf{susec:HFL}} or up to the largest one run in case the limit was not
    reached. The exact numbers are seen in \Arf{tab:HOHF, tab:HOHF3D}. The
    first test of the code was to reproduce the results of \cite{jorgenThesis,
    hfrefarticle} using
        \begin{equation}
            \Psi_T = \text{det}(\blds{\Phi})J_{\text{Pad\'e}}
            \label{eq:benchtrialwave}
        \end{equation}
    as the trial wavefunction where
        \begin{equation}
            \Phi_{ij} = \prod_d
            H_{n_{jd}}\left(\sqrt{\alpha\omega}x_{id}\right)\exp(-\frac{\alpha\omega}{2}x^2_{id})
        \end{equation}
    and
        \begin{equation}
            J_{\text{Pad\'e}} = \exp(\sumll{i<j} \frac{a_{ij}r_{ij}}{1+\beta
            r_{ij}}).
        \end{equation}
    Results are presented in \Arf{tab:HObench2D}

\subsection{Two-Dimensional Harmonic Oscillator}
    \input{text/tableHObench2D}

    \input{text/tHF2DVMCHO}

    The energies are consistently higher than with the wavefunction in
    \Arf{eq:benchtrialwave}. This only shows that the single-particle
    wavefunctions constructed from Hartre-Fock are actually not as good of a
    guess on the trial-wavefunction. Introducing a similar $\alpha$ parameter
    in the Hartree-Fock basis as
        \begin{equation}
            \psi^{\text{HF}}_p\left(\sqrt{\alpha\omega}\blds{r}\right) = \sum_l
            C_{lp} \psi^{\text{HO}}_l \left(\sqrt{\alpha\omega}\blds{r}\right)
            \label{eq:HFVARWAVE}
        \end{equation}
    reduces the energies further. The results are presented in
    \Arf{tab:HFVAR2DVMCHO}. This is basically taking the constructed basis from
    the Hartree-Fock simulations and evaluation the function with a variational
    parameter inspired from the same approach as with the results in
    \Arf{tab:HObench2D}.
    
    \input{text/tHFVAR2DVMCHO}

    As mentioned, the energies are consistently lower, however they are not as
    low as with the results with \Arf{eq:benchtrialwave}(\Arf{tab:HObench2D}).
    This again means that for the harmonic oscillator case the optimal trial
    wavefunction is indeed not the Hartree-Fock basis, but the harmonic
    oscillator functions. The energies are however very close giving a good
    foundation for believing the code works properly as intended.

\subsection{Three-Dimensional Harmonic Oscillator}
    Here are the same results for the three-dimensional case.
    \input{text/tableHObench3D}
    \input{text/tHF3DVMCHO}
    \input{text/tHFVAR3DVMCHO}

    Again the energies are higher than with \Arf{eq:benchtrialwave}, meaning
    the same conclusion for the optimal wavefunction as with the
    two-dimensional case is still the case for the three-dimensional case as
    well.

\subsection{Two-Dimensional Double-Well}
   \input{text/tHF2DVMCDW} 
   \input{text/tHFVAR2DVMCDW} 

\subsection{Three-Dimensional Double-Well}
   \input{text/tHF3DVMCDW} 
   \input{text/tHFVAR3DVMCDW} 

\subsection{Densities}
    The resulting one-body radial densities described in \Arf{sec:densityMat}
    are presented here.

\subsubsection{Harmonic Oscillator}
    \twofigure{{text/oneBodyFigs/HFHOVAROnebody/w0.28_D2_N2_HFHO_oneBody_rmax8.0.txt}.pdf}{$N=2$, $\omega=0.28$.}{HFHOOneBodyN20w028}{{text/oneBodyFigs/HFHOVAROnebody/w0.28_D2_N20_HFHO_oneBody_rmax10.0.txt}.pdf}{$N=20$, $\omega=0.28$.}{HFHOOneBodyN200w28}{One-Body density for the two dimensional harmonic oscillator potential using a Hartree-Fock basis with Pad\'e function and $\alpha$ parameter.}
    \twofigure{{text/oneBodyFigs/HFHOVAROnebody/w1.0_D2_N2_HFHO_oneBody_rmax4.0.txt}.pdf}{$N=2$, $\omega=1.0$.}{HFHOOneBodyN2w1}{{text/oneBodyFigs/HFHOVAROnebody/w1.0_D2_N20_HFHO_oneBody_rmax6.0.txt}.pdf}{$N=20$, $\omega=1.0$.}{HFHOOneBodyN200w1}{One-Body density for the two dimensional harmonic oscillator potential using a Hartree-Fock basis with Pad\'e function and $\alpha$ parameter.}
    \twofigure{{text/oneBodyFigs/HFHO3DVAROnebody/w0.1_D3_N2_HFHO_rmax10.0.txt}.pdf}{$N=2$, $\omega=0.1$.}{HFHO3DOneBodyN2w01}{{text/oneBodyFigs/HFHO3DVAROnebody/w0.1_D3_N8_HFHO_rmax14.0.txt}.pdf}{$N=20$, $\omega=0.1$.}{HFHOOneBodyN80w01}{One-Body density for the three dimensional harmonic oscillator potential using a Hartree-Fock basis with Pad\'e function and $\alpha$ parameter.}
    \twofigure{{text/oneBodyFigs/HFHO3DVAROnebody/w1.0_D3_N2_HFHO_rmax2.0.txt}.pdf}{$N=2$, $\omega=1.0$.}{HFHO3DOneBodyN2w1}{{text/oneBodyFigs/HFHO3DVAROnebody/w1.0_D3_N8_HFHO_rmax3.5.txt}.pdf}{$N=20$, $\omega=1.0$.}{HFHOOneBodyN8w1}{One-Body density for the three dimensional harmonic oscillator potential using a Hartree-Fock basis with Pad\'e function and $\alpha$ parameter.}

\subsubsection{Double-Well}
