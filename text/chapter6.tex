%%%%%%%%%%%%%%%%%%%%%%%%%%%%%%%%%% Chapter 6 %%%%%%%%%%%%%%%%%%%%%%%%%%%%%%%%%%
\chapter{Results \label{chapter:6}}
    No amount of physics theory will ever be interesting if it was not backed
    up by some proof or atleast some results. \\
    In this chapter we present the results from the simulations of the quantum
    dot system in the harmonic oscillator system and the double-well system. We
    also present some results regarding the minimization scheme in the VMC
    method.

\section{Tweaks and Experimentation}
    With the minimization methods clarified and the function to be minimized
    outlined the actual minimization could start\footnote{With the methods at
    disposal the \txtit{real challenge} was yet to come...}. \\
    Within the minimization methods presented there are quite the number of
    constant parameters which needed to be set pre-hand. Often one finds these
    some-what heuristically with a close eye on the function to be minimized.
    This approach is the one we used in the VMC method, with the most memorable
    one being the parameters set in the simulated annealing method.

\subsection{Simulated Annealing}
    With the simulated annealing the idea was to use it as a sort of
    "thermalization", that is to run it initially in hope that it finds the
    valley within the function-mesh in which the global minima lies. \\
    We tried anything from 

\section{Hartree-Fock}
\subsection{Harmonic Oscillator}
    \input{text/t1}
    \input{text/t3D}

\subsection{Double-Well\label{sec:dwbimp}}
    For the double-well we used the larges basis-set from the Hartre-Fock
    calculations, which was $120$ spacial functions. Meaning the sum over $l$
    in \Arf{eq:dwexpansiondef} runs up to $120$ while the number of
    basisfunctions used is choosen just as usual. Keep in mind that the
    degeneration is lifted in the double-well system. The resulting energies
    for are given in \Arf{tab:HODW2D, tab:HODW3D}.
    \input{text/t2DDW}
    \input{text/t3DDW}

\section{VMC}
    Here are the results for the quantum-dot simulations with the variational
    method both with and without a Hartre-Fock basis. We only used a basis as
    large as the one needed to reach the Hartree-Fock limit\footnote{See
    \Arf{susec:HFL}} or up to the largest one run in case the limit was not
    reached. The exact numbers are seen in \Arf{tab:HOHF, tab:HOHF3D}. The
    first test of the code was to reproduce the results of \cite{jorgenThesis,
    hfrefarticle} using
        \begin{equation}
            \Psi_T = \text{det}(\blds{\Phi})J_{\text{Pad\'e}}
            \label{eq:benchtrialwave}
        \end{equation}
    as the trial wavefunction where
        \begin{equation}
            \Phi_{ij} = \prod_d
            H_{n_{jd}}\left(\sqrt{\alpha\omega}x_{id}\right)\exp(-\frac{\alpha\omega}{2}x^2_{id})
        \end{equation}
    and
        \begin{equation}
            J_{\text{Pad\'e}} = \exp(\sumll{i<j} \frac{a_{ij}r_{ij}}{1+\beta
            r_{ij}}).
        \end{equation}
    Results are presented in \Arf{tab:HObench2D}
    \input{text/tableHObench2D}

    \input{text/tHF2DVMCHO}
    The energies are consistently higher than with the wavefunction in
    \Arf{eq:benchtrialwave}. This only shows that the single-particle
    wavefunctions constructed from Hartre-Fock are actually not as good of a
    guess on the trial-wavefunction. We will see that introducing a similar
    $\alpha$ parameter in the Hartree-Fock basis reduces the energies further.
