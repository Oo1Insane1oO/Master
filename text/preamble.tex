\documentclass[a4paper, hidelinks, 10pt]{book}
\usepackage[utf8]{inputenc}
\usepackage[T1]{fontenc}
\usepackage[expert]{mathdesign}
\usepackage{listings}
\usepackage[pdftex]{graphicx}
\usepackage{color}
\usepackage{mathtools}
\usepackage{amssymb}
\usepackage[margin=0.7in]{geometry}
\usepackage{hyperref}
\usepackage{subcaption}
\usepackage{float}
\usepackage{caption}
\usepackage{algpseudocode}
\usepackage{scrextend}
\usepackage{ragged2e}
\usepackage{multirow}
\usepackage{array}
\usepackage{physics}
\usepackage{enumitem}
\usepackage{varwidth}
\usepackage{tikz}
\usepackage{etoolbox}
\usepackage{fancyhdr}
%\usepackage{fouriernc}

%specific reused text
\newcommand{\mdate}{May 2018}
\newcommand{\mtitle}{Work in progress}
\newcommand{\mauthor}{Alfred Alocias Mariadason}
\newcommand{\massignn}{.}

\pagestyle{fancy}
\fancyhf{}
% \fancyhead[LO, RE]{\small\leftmark}
\lhead{\thesection}
\chead{}
\rhead{\small{\thechapter}}
% \lfoot{}
\cfoot{\thepage}
% \rfoot{}

%renew title numbering
% \renewcommand{\thechapter}{\arabic{chapter}}
\renewcommand{\thesection}{\arabic{section}}
\renewcommand{\thesubsection}{\thesection.\arabic{subsection}}

%title settings
% \renewcommand{\headrulewidth}{0pt}
\renewcommand{\chapterautorefname}{chapter}
\renewcommand{\sectionautorefname}{section}
\renewcommand{\subsectionautorefname}{section}
\renewcommand{\equationautorefname}{equation}
\renewcommand{\figureautorefname}{figure}
\renewcommand{\tableautorefname}{table}
\captionsetup{compatibility=false}

\patchcmd{\smallmatrix}{\thickspace}{\kern1.3em}{}{}

\definecolor{codegreen}{rgb}{0,0.6,0}
\definecolor{codegray}{rgb}{0.3,0.3,0.3}
\definecolor{codepurple}{rgb}{0.58,0,0.82}
\definecolor{backcolour}{rgb}{0.95,0.95,0.92}
\lstdefinestyle{mystyle}{
        backgroundcolor=\color{backcolour},
        commentstyle=\color{codegreen},
        keywordstyle=\color{magenta},
        numberstyle=\tiny\color{codegray},
        stringstyle=\color{codepurple},
        basicstyle=\footnotesize,
        breakatwhitespace=false,
        breaklines=true,
        captionpos=b,
        keepspaces=true,
        numbers=left, 
        numbersep=4pt, 
        showspaces=false, 
        showstringspaces=false,
        showtabs=true, 
        tabsize=2
}
\lstset{style=mystyle}

\hypersetup{
    colorlinks=true,
    linkcolor=black,
    filecolor=magenta,
    urlcolor=blue,
}
\urlstyle{same}

\newcommand{\onefigure}[4]{
    \begin{figure}[H]
        \centering
        \textbf{{#1}}\\
        \includegraphics[scale=0.65]{{#2}}
        \caption{{#3}}
        \label{fig:#4}
    \end{figure}
    \justifying
} %one figure {filename}{caption}
\newcommand{\twofigure}[7]{
    \begin{figure}[H]
        \centering
        \begin{subfigure}[b!]{0.49\textwidth}
            \centering
            \includegraphics[width=\textwidth]{{#1}}
            \caption{{#2}}
            \label{subfig:#3}
        \end{subfigure}
        \begin{subfigure}[b!]{0.49\textwidth}
            \centering
            \includegraphics[width=\textwidth]{{#4}}
            \caption{{#5}}
            \label{subfig:#6}
        \end{subfigure}
        \caption{#7}
        \justify
    \end{figure}
} %two figure one-line {title}{file1}{caption1}{file2}{caption2}

\newcommand{\prtl}{\mathrm{\partial}} %reduce length of partial (less to write)
\NewDocumentCommand{\prd}{m O{} O{}}{\frac{\prtl^{#3}{#2}}{\prtl{#1}^{#3}}}
\newcommand{\vsp}{\vspace{0.2cm}} %small vertical space
\newcommand{\txtit}[1]{\textit{{#1}}} %italic text
\newcommand{\blds}[1]{\boldsymbol{{#1}}} % better bold in mathmode (from amsmath)
\newcommand{\bigO}{\mathcal{O}} %nice big O
\newcommand{\me}{\mathrm{e}} %straight e for exp
\newcommand{\md}{\mathrm{d}} %straight d for differential
\newcommand{\mRe}[1]{\mathrm{Re}\left({#1}\right)}%nice real
\newcommand{\munit}[1]{\;\ensuremath{\, \mathrm{#1}}} %straight units in math
\newcommand{\Rarr}{\Rightarrow} %reduce lenght of Rightarrow (less to write)
\newcommand{\rarr}{\rightarrow} %reduce lenght of rightarrow (less to write)
\newcommand{\ecp}[1]{\left< {#1} \right>} %expected value
\newcommand{\urw}{\uparrow} % up arrow
\newcommand{\drw}{\downarrow} % up arrow
\newcommand{\pt}[1]{\textbf{\txtit{#1}}\justify}
\newcommand{\infint}{\int\limits^{\infty}_{-\infty}}
\newcommand{\oinfint}{\int\limits^{\infty}_0}
\newcommand{\sint}{\int\limits^{2\pi}_0\int\limits^{\pi}_0\oinfint}
\newcommand{\arcsinh}[1]{\text{arcsinh}\left(#1\right)}
\newcommand{\I}{\scalebox{1.2}{$\mathds{1}$}}
\newcommand{\veps}{\varepsilon} %\varepsilon is to long :P

\newcommand\numberthis{\addtocounter{equation}{1}\tag{\theequation}}

\makeatletter
% define a macro \Autoref to allow multiple references to be passed to \autoref
\newcommand\Autoref[1]{\@first@ref#1,@}
\def\@throw@dot#1.#2@{#1}% discard everything after the dot
\def\@set@refname#1{%    % set \@refname to autoefname+s using \getrefbykeydefault
    \edef\@tmp{\getrefbykeydefault{#1}{anchor}{}}%
    \def\@refname{\@nameuse{\expandafter\@throw@dot\@tmp.@autorefname}s}%
}
\def\@first@ref#1,#2{%
  \ifx#2@\autoref{#1}\let\@nextref\@gobble% only one ref, revert to normal \autoref
  \else%
    \@set@refname{#1}%  set \@refname to autoref name
    \@refname~\ref{#1}% add autoefname and first reference
    \let\@nextref\@next@ref% push processing to \@next@ref
  \fi%
  \@nextref#2%
}
\def\@next@ref#1,#2{%
   \ifx#2@ and~\ref{#1}\let\@nextref\@gobble% at end: print and+\ref and stop
   \else, \ref{#1}% print  ,+\ref and continue
   \fi%
   \@nextref#2%
}
\makeatother
