%%%%%%%%%%%%%%%%%%%%%%%%%%%%%%%%% Appendix B %%%%%%%%%%%%%%%%%%%%%%%%%%%%%%%%%%

\chapter{\label{appendix:B}}

% \section{Integral-elements in Least-Squares Fitting}
%     The integral elements are as follows
%         \begin{align}
%             X_{ij} &= \int g_i(x)g_j(x) \md x \nonumber \\
%             &= \int x^i N_i N_j \me^{\frac{\omega}{2} x^2} x^j
%             \me^{\frac{\omega}{2} x^2} \md x
%         \end{align}
%     and
%         \begin{align}
%             b_i &= \int g_i(x)f(x) \md x \nonumber \\
%             &= \int N_i x^i \me^{\frac{\omega}{2} x^2}
%             \left(2^nn!\sqrt{\frac{\pi}{\omega}}\right)^{-\frac{1}{2}}
%             H_n(\sqrt{\omega}x) \me^{\frac{\omega}{2} x^2} \md x
%         \end{align}
%     We dont contract the expressions further in favor of reusing the
%     normalization factors.

\section{Derivative of Energy and Variance}
    The derivative with respect to a general variational parameter $\alpha$ of
    the energy expectation value $\ecp{E}$ is
        \begin{align}
            \prd{\alpha}[\ecp{E}] &= 2\ecp{\frac{E_L}{\psi}\prd{\alpha}[\psi]}
            + \ecp{\prd{\alpha}[E_L]} -
            2\ecp{\frac{\ecp{E}}{\psi}\prd{\alpha}[\psi]} \nonumber \\
            &= 2\left(\ecp{\frac{E_L}{\psi}\prd{\alpha}[\psi]} -
            \ecp{E}\ecp{\frac{1}{\psi}\prd{\alpha}[\psi]}\right)
        \end{align}
    by using the hermiticity of the Hamiltonian (contained in the local energy
    $E_L$). \\
    The derivative of the variance is >> REF THIS STARTER <<
        \begin{align}
            \prd{\alpha}[\sigma^2] &= 2\ecp{\left(\prd{\alpha}[E_L] -
            \prd{\alpha}[\ecp{E}]\right) \left(E_L - \ecp{E}\right)} \nonumber
            \\
            &= 2\left(\ecp{E_L\prd{\alpha}[E_L]} -
            \ecp{E}\ecp{\prd{\alpha}[E_L]} +
            \ecp{\prd{\alpha}[\ecp{E}]\left(\ecp{E} - E_L\right)}\right)
            \nonumber \\
            &= 2\ecp{E}\left(\prd{\alpha}[\ecp{E}] - 1\right)
        \end{align}

\section{Derivatives of Hermite Functions}
    The gradient of Hermite functions on the form
        \begin{equation}
            \psi_n(r) = \prod_d \psi_{n_d}(x_d) = \prod_d
            H_{n_d}(\sqrt{\omega}x_d)\me^{-\frac{\omega}{2}x^2_d}
        \end{equation}
    is
        \begin{align}
            \nabla \psi_n(r) &= \sum_d\hat{e} \prod_{d'\neq d} \psi_{n_{d'}}
            \prd{x_d}[\psi_{n_d}] \nonumber \\
            &= \sum_d\hat{e} \prod_{d'\neq d} \psi_{n_{d'}} \psi_{n_d}
            \sqrt{\omega} \left(\prd{u}[H_{n_d}]\frac{1}{H_{n_d}} - x_d\right)
            \nonumber \\
            &= \psi_n\sqrt{\omega}\sum_d \hat{e}
            \left(2n_d\frac{H_{n_d-1}(\sqrt{\omega}x_d)}
            {H_{n_d}(\sqrt{\omega}x_d)} - x_d\right)
        \end{align}
    and the Laplacian follows
        \begin{align}
            \nabla^2 \psi_n(r) &= \sum_d \prod_{d'\neq d} \psi_{n_{d'}}
            \prd{x_d}[\psi_{n_d}][2] \nonumber \\
            &= \psi_n\omega\sum_d
            \left(\frac{\left(4n_d(n_d-1)H_{n_d-2}(\sqrt{\omega}x_d) -
            \sqrt{\omega}x_dH_{n_d-1}(\sqrt{\omega}x_d)\right)}
            {H_{n_d}(\sqrt{\omega}x_d)} + \omega x^2_d - 1\right)
        \end{align}
    The derivative with respect to the variational parameter $\alpha$ is
        \begin{align}
            \prd{\alpha}[\psi_n] &= \prd{\alpha} \prod_d
            H_{n_d}(\sqrt{\omega}x_d)\me^{-\frac{\omega}{2}x^2_d} \nonumber \\
            &= \sum_d \prod_{d'\neq d} \psi_{n_{d'}}(x_{d'})
            \prd{\alpha}[\psi_{n_d}(x_d)] \nonumber \\
            &= \psi_n(r)\sum_d \omega x_d\left(\frac{n_d}{\sqrt{\alpha\omega}}
            \frac{H_{n_d-1}\left(\sqrt{\alpha\omega}x_d\right)}
            {H_{n_d}\left(\sqrt{\alpha\omega}x_d\right)} - \frac{\omega
            x^2_d}{2}\right) \nonumber \\
            &= \psi_n(r)\left(\sqrt{\frac{\omega}{\alpha}} \sum_d x_dn_d
            \frac{H_{n_d-1}\left(\sqrt{\alpha\omega}x_d\right)}
            {H_{n_d}\left(\sqrt{\alpha\omega}x_d\right)} -
            \frac{\omega}{2}r^2\right)
        \end{align}

\section{Derivatives of Pad\'e-Jastrow Function}
    Given the Pad\'e-Jastrow function
        \begin{equation}
            J = \exp\left(\sum\limits_{i<j}f_{ij}\right) ,\indent f_{ij} =
            \frac{a_{ij}r_{ij}}{1 + \beta r_{ij}}
        \end{equation}
    The general expression for the gradient and Laplacian with respect to
    particle position $k$ is
        \begin{equation}
            \begin{aligned}
                \nabla_k J &= J\sum_{j\neq k} \frac{\blds{r}_{kj}}{r_{kj}}
                \prd{r_{kj}}[f_{kj}] \\
                \nabla^2_k J &= \frac{\left(\nabla_k J\right)^2}{J} +
                J\sum_{j\neq k} \left(\prd{r_{kj}}[f_{kj}] \frac{D-1}{r_{kj}} +
                \prd{r_{kj}}[f_{kj}][2]\right)
            \end{aligned}
        \end{equation}
    Notice that the sum with $j\neq k$ is only a sum over $j$ with $k$ fixed.
    The derivatives of $f$ are
        \begin{equation}
            \begin{aligned}
                \prd{r_{kj}}[f_{kj}] &= \frac{a_{kj}r^2}{\left(1 + \beta
                r_{kj}\right)^2} \\
                \prd{r_{kj}}[f_{kj}][2] &= -\frac{2a_{kj}\beta}{\left(1 + \beta
                r_{kj}\right)^3}
            \end{aligned}
        \end{equation}
    And the derivative with respect to the variational parameter $\beta$
        \begin{equation}
            \prd{\beta}[f_{kj}] = -\frac{a_{kj}r^2_{kj}}{\left(1 + \beta
            r_{kj}\right)^2}
        \end{equation}

%     Given the Pad\'e-Jastrow function
%         \begin{equation}
%             J = \exp\left(\sum\limits_{i<j}f_{ij}\right) ,\indent f_{ij} =
%             \frac{a_{ij}\sum\limits_l r^l_{ij}}{1 + \sum\limits_l \beta_l
%             r^l_{ij}}
%         \end{equation}
%     The general expression for the gradient and Laplacian with respect to
%     particle $k$ is
%         \begin{equation}
%             \begin{aligned}
%                 \nabla_k J &= J\sum_{j\neq k} \frac{\blds{r}_{kj}}{r_{kj}}
%                 \prd{r_{kj}}[f_{kj}] \\
%                 \nabla^2_k J &= \frac{\left(\nabla_k J\right)^2}{J} +
%                 J\sum_{j\neq k} \left(\prd{r_{kj}}[f_{kj}] \frac{D-1}{r_{kj}} +
%                 \prd{r_{kj}}[f_{kj}][2]\right)
%             \end{aligned}
%         \end{equation}
%     Notice that the sum with $j\neq k$ is only a sum over $j$ with $k$ fixed.
%     The derivatives of $f$ are (with the added conclusion that only
%     $a^{(1)}_{kj}$ is non-zero)
%         \begin{equation}
%             \begin{aligned}
%                 \prd{r_{kj}}[f_{kj}] &= \frac{a_{kj}\left(1 +
%                 \sumll{l}\beta_lr^l_{kj}\left(1-l\right)\right)}{\left(1 +
%                 \sumll{l}\beta_lr^l_{kj}\right)^2} \\
%                 \prd{r_{kj}}[f_{kj}][2] &=
%                 \frac{a_{kj}\left(\sumll{ll'}r^{l+l'}_{kj}\beta_l
%                 \left(\beta_{l'}\left(1-l^2\right) -
%                 \left(1-l'\right)\left(1+2l\right)\right) -
%                 \sumll{l}r^l_{kj}\beta_ll\left(l+1\right)\right)}{r_{kj}\left(1
%                 + \sumll{l}\beta_lr^l_{kj}\right)^3}
%             \end{aligned}
%         \end{equation}
%     The full derivation is as follows
%         \begin{align}
%             \prd{r_{kj}}[f_{kj}] &= \prd{r_{kj}} \left(\frac{\sum\limits_l
%             a^{(l)}_{kj}r^l_{ij}}{1 + \sum\limits_l \beta_l r^l_{ij}}\right)
%             \nonumber \\
%             &= \frac{\sum\limits_la^{(l)}_{kj}lr^{l-1}_{kj}\left(1 +
%             \sum\limits_l \beta_l r^l_{kj}\right) - \sum\limits_l
%             a^{(l)}_{kj}r^l_{kj}
%             \sum\limits_{l'}\beta_{l'}l'r^{l'-1}_{kj}}{\left(1 +
%             \sum\limits_l\beta_lr^l_{kj}\right)^2} \nonumber \\ 
%             &= \frac{\sum\limits_l a^{(l)}_{kj}lr^l_{kj} +
%             \sum\limits_{ll'}a^{(l)}_{kj}\beta_{l'}r^{l+l'}_{kj}
%             \left(l-l'\right)}{r_{kj}\left(1 +
%             \sum\limits_l\beta_lr^l_{kj}\right)^2} 
%         \end{align}
%     and
%         \newcommand{\akjl}{a^{(l)}_{kj}}
%         \newcommand{\akjll}{a^{(l')}_{kj}}
%         \begin{align}
%             \prd{r_{kj}}[f_{kj}][2] &= a_{kj}
%             \prd{r_{kj}}\left(\frac{\sum\limits_l a^{(l)}_{kj}lr^l_{kj} +
%             \sum\limits_{ll'}a^{(l)}_{kj}\beta_{l'}r^{l+l'}_{kj}
%             \left(l-l'\right)}{r_{kj}\left(1 +
%             \sum\limits_l\beta_lr^l_{kj}\right)^2}\right) \nonumber \\
%             &= \Biggm[\left(\sumll{l}\akjl l^2r^{l-1}_{kj} +
%             \sumll{ll'}\akjl\beta_{l'}r^{l+l'-1}_{kj}\left(l^2-l'^2\right)\right)
%             \left(r_{kj} \left(1 + \sum\limits_l\beta_lr^l_{kj}\right)^2\right)
%             \nonumber \\ &-\left(\left(1 +
%             \sum\limits_l\beta_lr^l_{kj}\right)^2 + 2r_{kj}\left(1 +
%             \sumll{l}\beta_lr^l_{kj}\right)\sumll{l}\beta_llr^{l-1}_{kj}\right)
%             \nonumber \\ &\times\left(\sumll{l}\akjl lr^l_{kj} +
%             \sumll{ll'}\akjl\beta_{l'}r^{l+l'}_{kj}\left(l-l'\right)\right)
%             \Biggm] \Biggm/ \left(r^2_{kj} \left(1 +
%             \sum\limits_l\beta_lr^l_{kj}\right)^4\right) \nonumber \\
%             &= \Biggm[\left(\sumll{l}\akjl l^2r^l_{kj} +
%             \sumll{ll'}\akjl\beta_{l'}r^{l+l'}_{kj}\left(l^2-l'^2\right)\right)
%             \left(1 + \sumll{l}\beta_lr^l_{kj}\right) \nonumber \\ &-\left(1 +
%             \sumll{l}\beta_lr^l_{kj} \left(1+2l\right) \right)
%             \left(\sum\limits_l\akjl lr^l_{kj} +
%             \sumll{ll'}\akjl\beta_{l'}r^{l+l'}_{kj}\left(l-l'\right)\right)
%             \Biggm] \nonumber \\ &\Biggm/ \left(r^2_{kj} \left(1 +
%             \sum\limits_l\beta_lr^l_{kj}\right)^3\right) \nonumber \\
%             &= \Biggm[\sumll{l}\akjl l^2r^l_{kj} + \sumll{ll'}\akjl
%             l^2\beta_{l'}r^{l+l'}_{kj} +
%             \sumll{ll'}\akjl\beta_{l'}r^{l+l'}_{kj}\left(l^2-l'^2\right) +
%             \sumll{ll'l''}\akjl\beta_{l'}\beta_{l''}r^{l+l'+l''}_{kj}
%             \left(l^2-l'^2\right) \nonumber \\ &- \Biggm(\sumll{l}\akjl
%             lr^l_{kj} +
%             \sumll{ll'}\akjl\beta_{l'}r^{l+l'}_{kj}\left(l-l'\right) +
%             \sumll{ll'}\akjll\beta_ll'r^{l+l'}_{kj}\left(1+2l\right) \nonumber
%             \\ &+ \sumll{ll'l''}\akjll\beta_l\beta_{l''}r^{l+l'+l''}_{kj}
%             \left(l'-l''\right) \left(1+2l\right) \Biggm] \Biggm/
%             \left(r^2_{kj} \left(1 +
%             \sum\limits_l\beta_lr^l_{kj}\right)^3\right) \nonumber \\
%             &= \frac{1}{r^2_{kj} \left(1 +
%             \sum\limits_l\beta_lr^l_{kj}\right)^3} \Biggm(\sumll{l}\akjl
%             r^l_{kj}l\left(l-1\right) \nonumber \\ &+ \sumll{ll'}
%             r^{l+l'}_{kj}\left(\akjl\beta_{l'}\left(2l^2-l'^2-l+l'\right) -
%             \akjll\beta_ll'\left(1+2l\right)\right) \nonumber \\ &+
%             \sumll{ll'l''}\beta_{l''}r^{l+l'+l''}_{kj}
%             \left(\akjl\beta_{l'}\left(l^2-l'^2\right) -
%             \akjll\beta_l\left(l'-l''\right)\left(1+2l\right)\right)\Biggm)
%         \end{align}
