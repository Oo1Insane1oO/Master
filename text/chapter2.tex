%%%%%%%%%%%%%%%%%%%%%%%%%%%%%%%%%% Chapter 2 %%%%%%%%%%%%%%%%%%%%%%%%%%%%%%%%%%
\chapter{}
    In this chapter we will address >> LIST METHODS << regarding computational
    quantum mechanics and further deepen into Hartree-Fock methods and
    Variational Monte Carlo method. Optimization of calculation is also given
    while structure of program is given in >> REF TO PROGRAM STRUCTURE CHAPTER
    <<. General statistical theory used is given in >> REF TO STATISTICS CHAPTER <<
\section{Quantum Monte Carlo}
    Quantum Monte Carlo, or QMC is a method for solving Schrödinger's equation
    by a statistical approach using so-called \txtit{Markov Chain} simulations
    (also called random walk). The nature of the wave function at hand is
    fundamentally a statistical model defined on a large configuration space
    with small areas of densities. The Monte Carlo method is perfect for
    solving such a system because of the non-homogeneous distribution of
    calculation across the space. An standard approach with equal distribution
    of calculation would then yield a rather poor result with respect to
    computation cost. \\
    We will in this chapter address the Metropolis algorithm which is used to
    create a Markov chain and derive the equations used in the variational
    method. \\
    The chapter will use \txtit{Dirac Notation} \cite{GriffQuan} and all
    equations stated assume atom units ($\hbar=m_e=e=4\pi\veps_0$) >> REF HERE
    ATOMIC UNITS <<.

    \subsection{The Variational Principle and Expectation Value of Energy}
        Given a Hamiltonian $\Ham$ and a trial wave function $\psiT$, the
        variational principle \cite{GriffQuan, NeOr} states that the
        expectation value of $\Ham$
            \begin{equation}
                E[\psi_T] = \ecp{\Ham} =
                \frac{\dinner{\psi_T}{\Ham}}{\pinner}
                \label{eq:ecpE}
            \end{equation}
        is an upper bound to the ground state energy
            \begin{equation}
                E_0 \leq \ecp{\Ham}
                \label{eq:ecpEBound}
            \end{equation}
        Now we can define our PDF as
            \begin{equation}
                P(\mb{R}) \equiv \frac{\abs{\psi_T}^2}{\pinner}
                \label{eq:PDFdef}
            \end{equation}
        and with a new quantity
            \begin{equation}
                E_L(\mb{R};\mb{\alpha}) \equiv \frac{1}{\psiT}\Ham\psiT
                \label{eq:ELdef}
            \end{equation}
        the so-called local energy, we can rewrite \Arf{eq:ecpE} as
            \begin{equation}
                E[\psiT] = \ecp{E_L}
            \end{equation}
        The idea now is to find the lowest possible energy by varying a set of
        parameters $\mb{\alpha}$. The expectation value itself is found with
        the Metropolis algorithm, see \Arf{susec:MHAlg}.

    \subsection{The Trial Wave Function}
        The trial wave function is generally an arbitrary choice specific for
        the problem at hand, however it is in most cases favorable to expand
        the wave function in the eigenbasis (eigenstates) of the Hamiltonian
        since they forma complete set. This can be expressed as
            \begin{equation}
                \psiT = \sum_i C_i\psi_i(\mb{R};\mb{\alpha})
            \end{equation}
        with the $\psi_i$'s are the eigenstates of the Hamiltonian.

    \subsection{Metropolis-Hastings Algorithm\label{susec:MHAlg}}
        The Metropolis algorithm bases itself on moves (also called
        transitions) as given in a Markov process. >> REF THIS HERE <<. This
        process is given by
            \begin{equation}
                w_i(t+\veps) = \sum_j\ufij{w}{i}{j}w_j(t)
            \end{equation}
        where $w(j\rarr i)$ is just a transition from state $j$ to state $i$.
        In order for the transition chain to reach a desired convergence while
        reversibility is kept, the well known condition for detailed balance
        must be fulfilled >> REF HERE DETAILED BALANCE <<. If detailed balance
        is true, then the following relations is true
            \begin{equation}
                w_i \ufij{T}{i}{j}\ufij{A}{i}{j} = w_j \ufij{T}{j}{i}\ufij{A}{j}{i}
                \Rarr \frac{w_i}{w_j} =
                \frac{\ufij{T}{j}{i}\ufij{A}{j}{i}}{\ufij{T}{i}{j}\ufij{A}{i}{j}}
                \label{eq:detailedBalance}
            \end{equation}
        We have here introduced two scenarios, the transition from
        configuration $i$ to configuration $j$ and the reverse process $j$ to
        $i$. Solving the acceptance $A$ for the two cases where the ratio in
        \ref{eq:detailedBalance} is either $1$(in which case the proposed state
        $j$ is accepted and transitions is made) and when the ratio is less
        then $1$. The Metropolis algorithm would in this case not automatically
        reject the latter case, but rather reject it with a proposed uniform
        probability. Introducing now a probability distribution function(PDF) $P$
        the acceptance $A$ can be expressed as
            \begin{equation}
                \ufij{A}{i}{j} =
                \text{min}\left(\frac{\ufij{P}{i}{j}}{\ufij{P}{j}{i}}
                \frac{\ufij{T}{i}{j}}{\ufij{T}{j}{i}} ,1\right)
                \label{eq:metropolisAcceptance}
            \end{equation}
        The so-called selection probability $T$ is defined specifically for
        each problem. For our case the PDF in question is the absolute square
        of the wave function and the selection $T$ is a Green's function given
        in >> REF GREENS <<. \\
        The algorithm itself would then be
            \begin{enumerate}[label=(\roman*)]
                \item Pick initial state $i$ at random.
                \item Pick proposed state at random in accordance to
                    $\ufij{T}{j}{i}$.
                \item Accept state according to $\ufij{A}{j}{i}$.
                \item Jump to step (ii) until a specified number of states have
                    been generated.
                \item Save the state $i$ and jump to step (ii).
            \end{enumerate}
