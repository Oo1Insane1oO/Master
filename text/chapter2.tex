%%%%%%%%%%%%%%%%%%%%%%%%%%%%%%%%%% Chapter 2 %%%%%%%%%%%%%%%%%%%%%%%%%%%%%%%%%%
\chapter{\label{chapter:2}}
    In this chapter we will list general theory regarding functions used in the
    methods mentioned in \Autoref{chapter:3} and show important properties used
    in later derivations.

\section{Variational Principle\label{sec:varPrinc}}
    The Variational principle states that for any normalized function $\psi$ in
    Hilbert Space >> REF HILBERT << with a Hermitian operator $H$ the minimum
    eigenvalue $E_0$ for $H$ has an upper-bound given by the expectation value
    of $H$ in the function $\psi$. That is
        \begin{equation}
            E_0 \leq \ecp{H} = \bra{\psi}H\ket{\psi} = \int \psi^{*}H\psi \md r
        \end{equation}
    See \cite{GriffQuan} for proof and more.

\section{Gaussian Type Orbitals}
    \txtit{Gaussian Type Orbitals} or GTO's are functions of the following form
    >> REF GTO here <<
        \begin{equation}
            G^{\alpha}_{i}(x,A) \equiv (x-A)^i\me^{-\alpha (x-A)^2}
            \label{eq:GTOdef1d}
        \end{equation}
    We call $\alpha$ for the scaling parameter and $i$ for the order of the
    GTO. The variable $A$ is where the function is centered. These are in many
    literatures referred to as \txtit{primitive gaussians}.
