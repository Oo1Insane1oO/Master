%%%%%%%%%%%%%%%%%%%%%%%%%%%%%%%%%% Chapter 2 %%%%%%%%%%%%%%%%%%%%%%%%%%%%%%%%%%
\chapter{\label{chapter:2}}
    In this chapter we will list general theory regarding functions used in the
    methods mentioned in \Autoref{chapter:3} and show important properties used
    in later derivations.

\section{Variational Principle\label{sec:varPrinc}}
    The Variational principle states that for any normalized function $\psi$ in
    Hilbert Space >> REF HILBERT << with a Hermitian operator $H$ the minimum
    eigenvalue $E_0$ for $H$ has an upper-bound given by the expectation value
    of $H$ in the function $\psi$. That is
        \begin{equation}
            E_0 \leq \ecp{H} = \bra{\psi}H\ket{\psi} = \int \psi^{*}H\psi \md r
        \end{equation}
    See \cite{GriffQuan} for proof and more.

\section{Lagrange Multiplier Method\label{sec:lagrange_multipliers}}
    See \cite{calcVar,calcVarSpring}. The optimization method of Lagrange
    multipliers maximizes(or minimizes) a function
    $f:\mathbb{R}^N\rightarrow\mathbb{R}$ with a constraint
    $g:\mathbb{R}^N\rightarrow\mathbb{R}$ We assume that $f$ and $g$ have
    continuous first derivatives in all variables(continuous first partial
    derivatives). \\
    Given the above we can define a so-called Lagrangian
    $\mathcal{L}$
        \begin{equation}
            \mathcal{L}[x_1,\dots,x_N,\lambda_1,\dots,\lambda_M] =
            f(x_1,\dots,x_N) - \lambda g(x_1,\dots,x_N)
            \label{eq:lagrangian}
        \end{equation}
    where the $\lambda$ is called a Lagrange-multiplier. We now state that if
    $f(x^0_1,\dots,x^0_N)$ is a maxima of $f(x_1,\dots,x_N)$ then there exists
    a Lagrange-multiplier $\lambda_0$ such that
    $(x^0_1,\dots,x^0_N,\lambda_0)$ is a stationary point
    for the Lagrangian. This then yields the $N+1$ Lagrange-equations
        \begin{align}
            \sum^N_{i=1} \frac{\prtl\mathcal{L}}{\prtl x_i} +
            \frac{\prtl\mathcal{L}}{\prtl \lambda} = 0
            \label{eq:lagrangeEQ}
        \end{align}
    to be solved for $x_1,\dots,x_N$ and $\lambda$.

\section{Gaussian Type Orbitals}
    \txtit{Gaussian Type Orbitals} or GTO's are functions of the following form
    >> REF GTO here <<
        \begin{equation}
            G^{\alpha}_{i}(x,A) \equiv (x-A)^i\me^{-\alpha (x-A)^2}
            \label{eq:GTOdef1d}
        \end{equation}
    We call $\alpha$ for the scaling parameter and $i$ for the order of the
    GTO. The variable $A$ is where the function is centered. These are in many
    literatures referred to as \txtit{primitive gaussians}.
